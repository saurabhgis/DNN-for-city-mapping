\documentclass{ctuthesis}

\usepackage{graphicx}

\ctusetup{
	xdoctype = M,
	xfaculty = F3,
	mainlanguage = english,
	titlelanguage = english,
	title-english = {Deep neural network for city mapping using Google Street View data},
	%	title-czech = {Sázení uranu},
	department-english = {Department of Cybernetics},
	author = {Varun Burde},
	supervisor = {Ing.Michal Reinštein,Ph.D.},
	supervisor-address = {E225b,\\ Karlovo nam. 13,\\ 121 35 Prague 2,\\ Czech Republic},
	month = 5,
	year = 2019,
}

\ctuprocess


\begin{abstract-english}
With the advancement computation power and large datasets finally made a huge improvement of Deep neural network leading many widespread applications. One of such application is solving computer vision problems like classification and segmentation. Also Competition like ImageNet\cite{StanfordVisionLab2015} Large Scale Visual Recognition Challenge, took the solution to next level, in some cases classification is better than Human. 
 
This report describes the evaluation of pre-trained deep neural network on Google StreetView Images\cite{Developers.google.com2018}. Pretrained model of Mask RCNN\cite{He2017} for matterport \cite{}.Implementation is done in Python\cite{PythonSoftwareFoundation2019} using Keras  and TesnsorFlow-GPU framework. User interface for the application execution, processing of the input images and visualization of the results is realized using Google Colab\cite{Colab.research.google.com} with repository in GIT\cite{Conservancy}.

A pipeline was created for the task, user provide the parameters like coordinates for the Google StreetView API \cite{Developers.google.com2018}. Python\cite{PythonSoftwareFoundation2019} script downloads the available images to that location then images were pre-processed to fit into classifier. Classification is done on the Images depending on the architecture of the Neural Network.A vectorized map is generated as a result of classified Images and put into map using Folium. 

Map is represented in various form as heat-map for better understanding of scenario. Object markers in map are the objects classified with the Mask RCNN .
	
\end{abstract-english}

\begin{abstract-czech}
	Rozvíjíme \ldots
	
\end{abstract-czech}

\begin{document}
		
\maketitle	

%!TEX ROOT=Thesis.tex

\chapter{Introduction}

\section{Aim and objective of project }

The aim is to design, implement and experimentally evaluate a deep neural network based solution for city mapping using Google Street View images. The proposed software solution should allow the user to request Google Street View imagery for any given location specified as geojson, perform analysis and feature extraction using deep neural network(s) and output vectorized description projected and visualized over an underlying map. User interface for the application execution, processing of the input images and visualization of the results should be realized using Google Colab to utilize Google TPUs. Existing pre-trained models should be explored first, thorough experimental evaluation on publicly available datasets should follow. Comparison with related state-of-the-art work is integral part of the work and should be presented in the final thesis. Recommendation: implementation should be done in Python, using Keras and TensorFlow frameworks.

\section{Overview of Project}

The project is software solution with the aim of achieving the task described above. There are few important part of the project which is implemented as a individual task in order to make it easy to debug. 

The core implementation of project is depends on the Google Street View \cite{Developers.google.com2018} imaginary. The API consist of helper function and parameter to get the images at certain location available in Google street view.Task was to Use API with Python \cite{PythonSoftwareFoundation2019} script to download the required images under the area by the coordinates provided .

Then this Images are processed with Deep Neural Network for classification and segmentation of objects, Mask R-CNN \cite{He2017} with pre-trained model consist of 82 classes from matterport \cite is used . The resulted Images are classified and segmented with the bounding box with there confidence. For purpose of project ,the confidence threshold is kept 0.7 for classification.

Experimentation is done to localized the object in map.Map is created with Folium \cite and represented in different ways ,the detected objects are put with different marker in the map and view is represented as heatmap to get better understanding of density of object and its type in the scenario











%!TEX ROOT=Thesis.tex

\chapter{SOTA}

Smart way of Google street view images\cite{Developers.google.com2018} as data set and using Machine learning algorithm specifically deep learning  to develop a software solution has been seen in last few years. some such application are \cite{Kita-wojciechowska} where images of house from GSV is annotated manually and and this data is use to predicts car accident of its resident. Other such application is taking use of GSV and satellite images to extract features like age,size and accessibility using Deep Neural Network to estimate the house prices.  

With the buzz of autonomous driving Layered Interpretation of Street View Images \cite{bibid} was developed in Mitsubishi Electric Research Labs with deep neural network on GSV images.In paper ,they proposed a layered street view model to encode both depth and semantic information on street view images for autonomous driving. They propose a 4-layer street view model, layers encode semantic classes like ground, pedestrians, vehicles, buildings, and sky in addition to the depths. The only input to our algorithm is a pair of stereo images. Deep neural network was used to extract the appearance features for semantic classes.

Another example of application developed with GSV and deep neural network is Building instance classification using street view images\cite{Kang2018} ,Land-use classification based on spaceborne or aerial remote sensing images has been extensively studied over the past decades. Proposed a general framework for classifying the functionality of individual buildings. The proposed method is based on Convolutional Neural Networks (CNNs) which classify facade structures from street view images, such as Google StreetView\cite{}, in addition to remote sensing images which usually only show roof structures. Geographic information was utilized to mask out individual buildings, and to associate the corresponding street view images. In addition, the method was applied to generate building classification maps on both region and city scales of several cities in Canada and the US

One of the such exampple which is similar to work done in the project is Automatic Discovery and Geotagging of Objects from Street View Imagery \cite{bibid} This paper describe about solution to localize the object from multiple view using geometry.To geolocate the object in image they developed Markov Field model to perform object triangulation.They use 2 SOTA  FCNN for semantic segmentation and monocular depth estimation.The geolocalization is done with Google street view images with Triangular based MRF model descriped in paper. End Result of the projects looks similar to the output of the resulting output but lacks the precision which is done with the help of depth image and Triangulation-based MRF model.The end result is a vectorized map with overlay of tags in map.



\input{theory}

%!TEX ROOT=Thesis.tex

\chapter{Methodology}

\section{hardware and software setup}

\subsection{hardware setup}
The uses the Tensorflow gpu backend along with Nvdia cuda dnn ver ..... and python as a scripting language. The local hardware used with GPU unit Nvdia Gtx 1060 and CPU unit - intel core i5 6300 CPU

\subsection{list of libraies used}
Many different python packages used for different purpose .

\subsubsection{Google street view}
Google street view library is used for the using google street view api which initiate the url request to the server with the parametes and developer key to download the images and metadata.

\begin{figure}
	\centering
	\includegraphics[width=0.7\linewidth]{picture/streetviewapi}
	\caption{street view structure}
	\label{fig:streetviewapi}
\end{figure}

Figure \ref{fig:list-of-classes} structure of how google street view api works

\subsubsection{Folium}
This package allow to create a world map place the marker of classified and detected objects into 2d world map with overlay of markers and also to create heatmaps 

\subsubsection{TensorFlow}
This package enables user to create high dimensional data type which can be efficiently use to perform high computation process on huge and dense data 

\subsubsection{Keras}
This package enable user to define the architecture of the Deep Neural network, defining the order of different layer there activation functions ,learning rate etc.

\section{Pretrained model details}
The architecture and pre-trained model use in this project is taken from Matterport  \cite{matterport_maskrcnn_2017}.The pre-trained model was trained on COCO \cite{DBLP:journals/corr/LinMBHPRDZ14} data set with 81 classes.

\begin{figure}
	\centering
	\includegraphics[width=0.7\linewidth]{"picture/list of classes"}
	\caption{List of classes}
	\label{fig:list-of-classes}
\end{figure}

Figure \ref{fig:list-of-classes} shows the list of classes which can be used with pre - trained model caption.

\section{Google street view}


\subsection{Basic overview}
Street View, by Google Maps, is a virtual representation of our surroundings on Google Maps, consisting of millions of panoramic images. Street View’s content comes from two sources - Google and contributors. Through our collective efforts, we enable people everywhere to virtually explore the world.\cite{Developers.google.com2018}

Images form street view was the major dataset for the project.Street view images are in form of 360 panorama,for the purpose of projects 640x640 image is used with field of view 90 to process the DNN efficiently.  


\subsection{Parameters in Street view }
Google street view api uses python sent a url request to server in order to download the images , request is in form of string made up of parameters needed to fulfill the query. server returns back with the metadata with the information of images such as date , location in form of latitude and lattitude , panorama id ,status file name. 

parameter string is in form of python dictionary with certain key like location , size ,heading , field of view, pitch and developer key . Developer key is one of the most important key in parameter to make this Google street api works. 

\begin{figure}
	\centering
	\includegraphics[width=0.7\linewidth]{picture/parametergsv}
	\caption{parameters of GSV images}
	\label{fig:parametergsv}
\end{figure}

Figure \ref{fig:parametergsv} Example of street view api parameters to showing apiargs dictionary with key and values .

\subsection{Selection of parameters}

Location describes the coordinates of location in form of numbers of latitude, longitude. Multiple location can be fitted in arguments. If no image is available then API will return a generic image with text “Sorry, we have no imagery here."
Size is the resolution of downloaded image. Highest resolution can be downloaded is 2048x2048 which is available with google premium account services. In project standard 640x640 image is used for the classification.

Heading indicated the compass heading of the camera, accepted values are 0 to 360. North is indicated by 0. seven different heading angles were chosen for project cause each angle to have something different scenario along with some similarity with adjacent image makes easy to detect the confused objects.

FOV is set to 90 for the project for optimum setting to avoid pixilation in case of zoom in and to less info per pixel in case of zoom out. For the privacy purpose the developer key is hidden.

Parameters were chosen on basis will give us the best images with orientation, less noise and perfect fit for our model to test. For ex – having pitch to -90 or 90 gives the images headed up to sky and bottom to floor.

\begin{figure}
	\centering
	\includegraphics[width=0.4\linewidth]{"picture/pitchwith 90"}
	\caption{pitch with 90}
	\label{fig:pitchwith 90}
\end{figure}

\begin{figure}
	\centering
	\includegraphics[width=0.4\linewidth]{"picture/pitchwith-90"}
	\caption{pitch with -90}
	\label{fig:pitchwith-90}
\end{figure}

Figure \ref{fig:pitchwith-90} and \ref{fig:pitchwith-90} Example images with pitch 90 and -90.


\section{Mask Rcnn}

\subsection{Description about Mask Rcnn}
Mask R-CNN\cite{He2017} is state of the art Deep neural network proposed by Facebook research scientist Kaiming HE in 2017 which can perform instance segmentation and object detection together.I can have several different backbone architecture like Inception V2 ,ResNet 50 ,Resnet101 and Inception-Rennet2.For project ,the pre-trained model on COCO dataset with Resnet 101 architecture was used.Mask R-CNN with Resnet-101-FPN backbone was able to outperform other state of the art network like MNC and FCIS winner of COCO 2015 and 2016 segmentation challenge. 

\subsection{How Its implement in python with Keras and tensorflow backend with citiation }
Implementation of architecture of Mask R-CNN was done in Keras framework and TensorFlow backend ,Its taken from the open source repository from Matterport \cite{matterport_maskrcnn_2017} .Implementation of Mask R-CNN is done in python and can be seen in script model.py ,It consist definition of all the layer and architecture of Mask R-CNN in TensorFlow.

\subsection{Visulization of output from Mask R-CNN}
Implementation of creation of masks and image processing is done with the Opencv \cite{Culjak2012}.The output from the results are in form of image with overlay of text(class name) with number(confidence of being of that class,Form 0 to 1 ,being 1 means full confidence) bounding box to localize the classified object in Image and mask with color to cover pixel wise segmentation of detected object in image.

\subsubsection{Parameters in Model}
With ResNet 101 as back bone architecture, model provide several other parameters tweak to fit for application .Some of the easily configurable parameters are Minimum confidence detection , learning momentum , learning rate .etc 

\begin{figure}
	\centering
	\includegraphics[width=0.7\linewidth]{"picture/config of mrcn"}
	\caption{Configurable parameter in Mask R-CNN}
	\label{fig:config-of-mrcn}
\end{figure}

Figure \ref{fig:config-of-mrcn} shows some of the configurable parameter of Mask R-CNN .Out of which number of classes depends on the pre trained model during testing. Minimum confidence decide creation of bounding box if the minimum 0.7 confidence is achieved at classification of that object. 

\subsection{Classes used in map}
As the end result of project results in heatmap of the objects detected in streets and the pre-trained model consist of several classes with doesnt show any significance to result ,rather can make some false results in order to avoid those. Only few classes were used for heatmap which describe the scenario of the area under test.

\begin{figure}
	\centering
	\includegraphics[width=0.7\linewidth]{picture/knife}
	\caption{wronf classification with good confidence}
	\label{fig:knife}
\end{figure}
Figure \ref{fig:knife} show example of class knife which is classified wrong and make less significance in street view images.

\subsection{some drawback of Mask R-CNN during classification}
misclassification and correct classification 
how I dealt with too classification and eliminate the wrong one which parameters I changed


\section{Google Colab}

\subsection{basic overview}

\subsection{advantages of using this}

\subsection{changes needed to implement this }

\subsection{how it improved the performance ,types of runtime}




\section{Folium}

\subsection{basic overview and its features}

\subsection{how i used it in project }

\subsection{classes and types i object I marked in the map and why did I skipped others}

\subsection{types of representation of maps }

will describe the representation in form of heat-map , markers and what different kind of Information I can get from different representation




\section{experimentation}

\subsection{use of depth images }

\subsection{difference in usage of depth in mapping and without depth}

\subsection{Algorithm about how the objects are localize in the 2d map}

\subsection{how i tackle with the localization of same object }

\subsection{algorithm to find the proper image under the polygon}
like why i need proper image and what will happen if i dont use proper images 



\input{Implementaion}

\input{exper}

%!TEX ROOT=Thesis.tex

\chapter{Conclusion}	

ghghjkghkgkh \cite{Chollet2017}
		
\chapter{Bibliography}

\bibliographystyle{ieeetr}	
\bibliography{thesis}

\end{document}

